% Options for packages loaded elsewhere
\PassOptionsToPackage{unicode}{hyperref}
\PassOptionsToPackage{hyphens}{url}
\PassOptionsToPackage{dvipsnames,svgnames,x11names}{xcolor}
%
\documentclass[
  10pt,
]{article}

\usepackage{amsmath,amssymb}
\usepackage{iftex}
\ifPDFTeX
  \usepackage[T1]{fontenc}
  \usepackage[utf8]{inputenc}
  \usepackage{textcomp} % provide euro and other symbols
\else % if luatex or xetex
  \usepackage{unicode-math}
  \defaultfontfeatures{Scale=MatchLowercase}
  \defaultfontfeatures[\rmfamily]{Ligatures=TeX,Scale=1}
\fi
\usepackage{lmodern}
\ifPDFTeX\else  
    % xetex/luatex font selection
\fi
% Use upquote if available, for straight quotes in verbatim environments
\IfFileExists{upquote.sty}{\usepackage{upquote}}{}
\IfFileExists{microtype.sty}{% use microtype if available
  \usepackage[]{microtype}
  \UseMicrotypeSet[protrusion]{basicmath} % disable protrusion for tt fonts
}{}
\makeatletter
\@ifundefined{KOMAClassName}{% if non-KOMA class
  \IfFileExists{parskip.sty}{%
    \usepackage{parskip}
  }{% else
    \setlength{\parindent}{0pt}
    \setlength{\parskip}{6pt plus 2pt minus 1pt}}
}{% if KOMA class
  \KOMAoptions{parskip=half}}
\makeatother
\usepackage{xcolor}
\setlength{\emergencystretch}{3em} % prevent overfull lines
\setcounter{secnumdepth}{-\maxdimen} % remove section numbering
% Make \paragraph and \subparagraph free-standing
\makeatletter
\ifx\paragraph\undefined\else
  \let\oldparagraph\paragraph
  \renewcommand{\paragraph}{
    \@ifstar
      \xxxParagraphStar
      \xxxParagraphNoStar
  }
  \newcommand{\xxxParagraphStar}[1]{\oldparagraph*{#1}\mbox{}}
  \newcommand{\xxxParagraphNoStar}[1]{\oldparagraph{#1}\mbox{}}
\fi
\ifx\subparagraph\undefined\else
  \let\oldsubparagraph\subparagraph
  \renewcommand{\subparagraph}{
    \@ifstar
      \xxxSubParagraphStar
      \xxxSubParagraphNoStar
  }
  \newcommand{\xxxSubParagraphStar}[1]{\oldsubparagraph*{#1}\mbox{}}
  \newcommand{\xxxSubParagraphNoStar}[1]{\oldsubparagraph{#1}\mbox{}}
\fi
\makeatother


\providecommand{\tightlist}{%
  \setlength{\itemsep}{0pt}\setlength{\parskip}{0pt}}\usepackage{longtable,booktabs,array}
\usepackage{calc} % for calculating minipage widths
% Correct order of tables after \paragraph or \subparagraph
\usepackage{etoolbox}
\makeatletter
\patchcmd\longtable{\par}{\if@noskipsec\mbox{}\fi\par}{}{}
\makeatother
% Allow footnotes in longtable head/foot
\IfFileExists{footnotehyper.sty}{\usepackage{footnotehyper}}{\usepackage{footnote}}
\makesavenoteenv{longtable}
\usepackage{graphicx}
\makeatletter
\def\maxwidth{\ifdim\Gin@nat@width>\linewidth\linewidth\else\Gin@nat@width\fi}
\def\maxheight{\ifdim\Gin@nat@height>\textheight\textheight\else\Gin@nat@height\fi}
\makeatother
% Scale images if necessary, so that they will not overflow the page
% margins by default, and it is still possible to overwrite the defaults
% using explicit options in \includegraphics[width, height, ...]{}
\setkeys{Gin}{width=\maxwidth,height=\maxheight,keepaspectratio}
% Set default figure placement to htbp
\makeatletter
\def\fps@figure{htbp}
\makeatother

\usepackage[margin=1in, top=0.6in]{geometry}
\usepackage{titlesec}
\titlespacing*{\title}{0pt}{0pt}{20pt}
\makeatletter
\@ifpackageloaded{caption}{}{\usepackage{caption}}
\AtBeginDocument{%
\ifdefined\contentsname
  \renewcommand*\contentsname{Table of contents}
\else
  \newcommand\contentsname{Table of contents}
\fi
\ifdefined\listfigurename
  \renewcommand*\listfigurename{List of Figures}
\else
  \newcommand\listfigurename{List of Figures}
\fi
\ifdefined\listtablename
  \renewcommand*\listtablename{List of Tables}
\else
  \newcommand\listtablename{List of Tables}
\fi
\ifdefined\figurename
  \renewcommand*\figurename{Figure}
\else
  \newcommand\figurename{Figure}
\fi
\ifdefined\tablename
  \renewcommand*\tablename{Table}
\else
  \newcommand\tablename{Table}
\fi
}
\@ifpackageloaded{float}{}{\usepackage{float}}
\floatstyle{ruled}
\@ifundefined{c@chapter}{\newfloat{codelisting}{h}{lop}}{\newfloat{codelisting}{h}{lop}[chapter]}
\floatname{codelisting}{Listing}
\newcommand*\listoflistings{\listof{codelisting}{List of Listings}}
\makeatother
\makeatletter
\makeatother
\makeatletter
\@ifpackageloaded{caption}{}{\usepackage{caption}}
\@ifpackageloaded{subcaption}{}{\usepackage{subcaption}}
\makeatother

\ifLuaTeX
  \usepackage{selnolig}  % disable illegal ligatures
\fi
\usepackage{bookmark}

\IfFileExists{xurl.sty}{\usepackage{xurl}}{} % add URL line breaks if available
\urlstyle{same} % disable monospaced font for URLs
\hypersetup{
  pdftitle={PPHA:30538 Final Project Writeup},
  pdfauthor={Cristian Bancayan, Claudia Felipe \& Sol Rivas Lopes},
  colorlinks=true,
  linkcolor={blue},
  filecolor={Maroon},
  citecolor={Blue},
  urlcolor={Blue},
  pdfcreator={LaTeX via pandoc}}


\title{\large PPHA:30538 Final Project Writeup}
\author{\normalsize Cristian Bancayan, Claudia Felipe \& Sol Rivas
Lopes}
\date{}

\begin{document}
\maketitle


\subsection{Motivation}\label{motivation}

Since their inception in Latin America in the 1990's, Conditional Cash
Transfers (CCT) have been the anti-poverty program of choice in the
region and in other developing countries. The programs' general
structure often entail an income subsidy given to poor families,
benefits which are tied to their children's school enrollment.

The success of these programs in increasing school enrollment and
mitigating poverty at the national level is well-documented. However,
fewer research has focused on the differential impacts of CCT programs
between urban and rural populations. In cases where families withdraw
children from school to supplement household income---a situation common
in both urban and rural areas---CCTs effectively address the issue by
providing income subsidies. However, if low school enrollment is due to
limited access to schools, which is more prevalent in rural areas with
inadequate infrastructure, CCTs alone may be an insufficient strategy to
combat intergenerational poverty and inequality.

Motivated by such discrepancies, this research project aims to
understand whether there is evidence of differential long-term impacts
of CCTs in education and quality of life outcomes in rural and urban
areas.

\begin{figure}[H]

{\centering \includegraphics[width=0.6\textwidth,height=\textheight]{Graphs/years_edu_all.png}

}

\caption{Years of Education over years}

\end{figure}%

The figure shows the aggregated median years of education in Brazil,
Chile, Mexico, Paraguay, and Peru. While there is a general upward
trend, urban populations have, on average, two more years of education
than rural populations. This pattern is consistent across all countries,
with disaggregated visualizations available in the Shiny dashboard.

\subsection{Data Source and Approach}\label{data-source-and-approach}

To answer the research question, this project used data compiled by the
Center for Distributive, Labor and Social Studies (CEDLAS), from the
National University of La Plata in Argentina, in partnership with the
World Bank. We retrieved datasets about education, infrastrcutre, and
housing, publicly available on their
\href{https://www.cedlas.econo.unlp.edu.ar/wp/en/estadisticas/sedlac/estadisticas/}{website}.

The cleaning process (\href{./cleaning_education.qmd}{education data}
and \href{./cleaning_infraestructure.qmd}{infrastructure data}) selected
the information depending on the survey methodologies for each country.
Then, for the analysis, we narrowed the datasets to include only the
target countries and variables of interest. Brazil, Chile, Mexico,
Paraguay, and Peru were selected because they had both (a) implemented a
CCT program and (b) available rural and urban disaggregated data.
Variables were refined to focus only on relevant outcomes. We considered
years of education to gauge urban and rural disparities broadly.
However, we focused on school enrollment for 6- to 12-year-olds and 13-
to 17-year-olds to better capture immediate impacts of CCTs.

To explore outcomes beyond education, we wanted to examine changes to
the quality of life before and after CCT implementation. Because health,
mental health and labor force participation have been widely studied, we
opted to delve into the quality of dwellings. This addresses a gap in
the literature and consideres that dwelling quality is less influenced
by systemic differences across countries. While health outcomes will
vary according to a country's public health and healthcare strucutre, a
safe, well-built house is likely to be consistent across all countires.
Furthermore, home improvement is an area where individuals have
significant autonomy and often allocate surplus income.

After cleaning and merging the data,
\href{./final_project_initial_analysis.qmd}{we created clear and
informative plots to visualize the data effectively}. We then conducted
simple regressions, correlation analyses, and t-tests to determine
whether the observed differences in the data were statistically
significant.

Given time limitations, the project did not aim to establish causal
relationships. All findings are based on descriptive and correlational
analyses, without robustness checks for causality. Nevertheless, the
insights gathered provide valuable descriptive information that can
inform future policy.

\subsection{Findings}\label{findings}

The main finding is that rural areas might lead the increase in school
enrollment associated with the implementation of CCT programs. The
average percentage-point increase in rural school enrollment in the
periods after the implementation of a CCT program relative to periods
before exceeds that of urban areas across all countries.

Such improvements are particularly meaningful for the 13- to 17-year-old
group, suggesting that CCTs played a key role in keeping teenagers in
school. This indicates that the subsidies likely offset or outweigh the
immediate income benefits of enterging the workforce before school
completion. Furthermore, these improvements highlight that interruptions
in education are primarily driven by the need to supplement family
income.

\begin{figure}

\begin{minipage}{0.50\linewidth}

\includegraphics[width=0.9\textwidth,height=\textheight]{Graphs/change_enrollment6_12yo.png}

\subcaption{\label{}Figure 2a}
\end{minipage}%
%
\begin{minipage}{0.50\linewidth}

\includegraphics[width=0.9\textwidth,height=\textheight]{Graphs/change_enrollment13_17yo.png}

\subcaption{\label{}Figure 2b}
\end{minipage}%

\end{figure}%

Figures 2a and 2b depict the average percentage point increase in school
enrollment following the implementation of a CCT program in a given
country, for 6- to 12-year-olds and 13- to 17-year-olds, respectively.
Rural areas across all countries consistently exhibit greater enrollment
increases compared to urban areas.

We also explored how higher enrollment rates could translate into better
homes. Running a simple correlation, we found that in Brazil, Chile, and
Mexico, years of education and dwelling quality are highly inversely
correlated, with coefficients exceeding 0.9. In Paraguay, the
correlation is initially low, but after dropping two outliers, the same
linear relationship holds. Peru is the only country where these two
measures appear to be uncorrelated. These findings therefore suggest
that CCT programs indirectly and negatively might affect the share of
rural populations living in poor dwellings.

\begin{figure}[H]

{\centering \includegraphics[width=0.6\textwidth,height=\textheight]{Graphs/corr_edu_dwelling.png}

}

\caption{Correlation Dwellings and Years of Education}

\end{figure}%

Figure 3 shows the correlation between years of education and the
average share of the population living in poor dwellings in rural areas,
disaggregated by country. :::

This intuition is confirmed by visualizing how the share of poor
dwellings decreases after the implementation of CCT programs in each of
those 4 countries, even when effects are not immediate. A more
sophisticated identification strategy, such as a differences in
differences design, is needed to explore causality without confounding
factors. Still, these preliminary graphs are an optimistic start for
future research on the topic.

For additional plots and comparisons, a \href{./shiny-app/app.py}{Shiny
dashboard} allows the user to allows explore trends for the outcomes
preivously mentioned across the five different countries analyzed.

\subsection{Policy Implications}\label{policy-implications}

Unlike our initial expectation, CCTs played a key role in increasing
enrollment rates, especially in rural areas. The long-term benefits go
beyond educational and financial gains, and could also possibly extend
to housing quality, although further research is needed in this respect.

Although remaining disparities indicate a need for targeted infrastrcure
and education policies to better serve rural populations, these findings
reinforce the success of CCT programs as a transformative force in
breaking the cycle of poverty across both urban and rural landscapes.




\end{document}
